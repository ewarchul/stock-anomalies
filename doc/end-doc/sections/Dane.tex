\section{Analiza danych}
\label{section:analiza}

Sekcja ta poświęcona jest analizie danych. Opis oraz krótka charakterystyka danych, których używaliśmy w projekcie znajduje się w dokumentacji wstępnej \cite{Dzienisz}. 

W ramach przeprowadzonej analizy skupiliśmy się na następujących zagadnieniach:

\begin{itemize}

  \item zbadaliśmy rozkład stosowanych danych. Zastosowane przez nas modele w dużej mierze były wolne od założeń odnośnie rozkładu danych (ang. distribution free), ale znajomość
  rozkładu stanowi istotną wartość dodaną analizy. Dzięki niej byliśmy w stanie ocenić częstość występowania anomalii odległych od wartości modalnej rozkładu. 
  \item analiza rozkładu została również roszerzona o miary rozkładu jak mediana, kurtoza czy skośność. 
  \item w związku z faktem, że analizowane dane stanowią szereg czasowy istotne było zbadanie autokorelacji. Na jej podstawie uzskaliśmy pewne pojęcie o sezonowności
  analizowanych szeregów.
  \item wykorzystaliśmy również wykładnik Hurst'a jako uzupełnienie analizy autokorelacji szeregu czasowego. Wykładnik ten jest definiowany w następujący sposób:
    \begin{equation}
        \mathcal{H}(t) = \mathcal{E}\frac{R(t)}{S(t)}
    \end{equation}
    prz czym $R(t)$ reprezentuje odchylenie od średniej $t$ początkowych wartości szeregów, a $S(t)$ jest ich odchyleniem standardowym. 
    Jeśli wartości $\mathcal{H}$ znajdują się w zbiorze $(0.5, 1)$, to spodziewać się można, że duże wartości szeregu czasowego nie będą pojawiały sie w odosobnieniu. Innymi słowy, 
    pojawienie się dużych wartości szeregu czasowego będzie związane z przesunięciem średniej w większym oknie czasowym.
\end{itemize}

\subsection{Imputacja anomalii}
\label{subsection:imputacja}


\subsection{Dane WoW}
\label{subsection:dane-wow}


\subsection{Dane WiG}
\label{subsection:dane-wig}

